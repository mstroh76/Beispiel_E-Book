\chapter{Verlinkungen}

\section{Kapitel}

Im Kapiel \ref{chap:Tabelle} \titleref{chap:Tabelle} kann die Darstellung von Tabellen �berpr�ft werden.  

\section{Unterkapitel}

Im Unterkapitel \ref{sec:Nummeriert} \titleref{sec:Nummeriert} wird die Formatierung einer nummerierten Aufz�hlung dargestellt.

\section{Grafik}

Raspberry Pi als JPG-Grafik siehe Abbildung \ref{fig:Raspberry_Pi_Foto} \titleref{fig:Raspberry_Pi_Foto}.\\
Raspberry Pi als PNG-Grafik siehe Abbildung \ref{fig:Raspberry_Pi_Clipart} \titleref{fig:Raspberry_Pi_Clipart}.\\


\section{Quellcode}

Kompiliert man die Datei \ref{lst:HelloWord_c} \titleref{lst:HelloWord_c} mit gcc erh�lt man eine ausf�hrbare Datei.\\
Das Skript von \ref{lst:HelloWord_sh} \titleref{lst:HelloWord_sh} kann direkt ausgef�hrt werden.


\section{Tabelle}

Die Tabelle \ref{tab:schmall} \titleref{tab:schmall} kann komplett am Bildschirm des Kindle Paperwhite dargestellt werden.

\section{Formel}

Die Formel \ref{eq:Bruch} \titleref{eq:Bruch} kann nur als Grafik dargestellt werden.