\chapter{Mathematik (Formeln)}

\section{Formeln direkt darstellbar}

\begin{Large}
$ U_{2} = U / ( R_{1} + R_{2} ) \cdot R_{2} $ \\

$ P = I_{f}^{2} \cdot R $ \\

$ s_{xy} = \int_{x}^{y} (a \cdot t + V_{0}) \cdot dt $

\end{Large}

\section{Formeln als Grafik}

\paragraph{Bruch}~\\

\begin{Large}
$ I_{f} = \frac{U_{f}}{R} \label{eq:Bruch} $ \\
\end{Large}


\paragraph{Wurzel}~\\

\begin{Large}
$ c=\sqrt[]{a^{2}+b^{2}} \label{eq:Wurzel} $ \\
\end{Large}


\paragraph{Explizit als Grafik formatiert}~\\

\begin{Large}
$$ U_{2} = U / ( R_{1} + R_{2} ) \cdot R_{2} $$\\
\end{Large}


\section{Griechische Buchstaben}

$\Gamma \dots$ Gamma\\
$\Delta \dots$ Delta\\
$\Theta \dots$ Theta\\
$\Lambda \dots$ Lambda\\
$\Xi \dots$ Xi\\
$\Pi \dots$ Pi\\
$\Sigma \dots$ Sigma\\
$\Upsilon \dots$ Sigma\\
$\Phi \dots$ Ypsilon\\
$\Psi \dots$ Psi\\
$\Omega \dots$ Omega\\
$\alpha \dots$ Alpha\\
$\beta \dots$ Beta\\
$\gamma \dots$ Gamma\\
$\delta \dots$ Delta\\
$\epsilon \dots$ Epsilon\\
$\varepsilon \dots$ Epsilon (varepsilon)\\
$\zeta \dots$ Zeta\\
$\eta \dots$ Eta\\
$\theta \dots$ Theta\\ 	
$\vartheta \dots$ Theta (vartheta)\\
$\iota \dots$ Iota\\
$\kappa \dots$ Kappa\\
$\lambda \dots$ Lambda\\ 	
$\mu \dots$ My\\
$\nu \dots$ Ny\\
$\xi \dots$ Xi\\
$\pi \dots$ Pi\\
$\varpi \dots$ Pi (varpi)\\
$\rho \dots$ Rho\\
$\varrho \dots$ Rho (varrho)\\
$\sigma \dots$ Sigma\\
$\varsigma \dots$ Sigma (varsigma)\\
$\tau \dots$ Tau\\
$\upsilon \dots$ Ypsilon\\ 	
$\phi \dots$ Phi\\
$\varphi \dots$ Phi (varphi)\\
$\chi \dots$ Chi\\
$\psi \dots$ Psi\\
$\omega \dots$ Omega\\


\section{Operatoren}

%https://de.wikipedia.org/wiki/Liste_mathematischer_Symbole

$\le \dots$ Gr��er als oder gleich\\
$\ge \dots$ Kleiner als oder gleich\\
$\approx \dots$ Ungef�hr\\
$\neq \dots$ Ungleich\\
$\ll \dots$ Wesentlich Gr��er\\
$\gg \dots$ Wesentlich Kleiner\\
$\pm \dots$ Plusminus\\
$\mp \dots$ Minusplus\\
$\times \dots$ Multiplikator\\ 
$\cdot \dots$ Multiplikator\\
